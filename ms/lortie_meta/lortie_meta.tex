%% Submissions for peer-review must enable line-numbering
%% using the lineno option in the \documentclass command.
%%
%% Preprints and camera-ready submissions do not need
%% line numbers, and should have this option removed.
%%
%% Please note that the line numbering option requires
%% version 1.1 or newer of the wlpeerj.cls file, and
%% the corresponding author info requires v1.2

\documentclass[fleqn,10pt]{wlpeerj} % for preprint submissions

% ZNK -- Adding headers for pandoc

\setlength{\emergencystretch}{3em}
\providecommand{\tightlist}{
\setlength{\itemsep}{0pt}\setlength{\parskip}{0pt}}
\usepackage{lipsum}
\usepackage[unicode=true]{hyperref}
\usepackage{longtable}



\usepackage{lipsum}

\title{A contrast of meta and metafor packages for meta-analyses in R}

\author[1]{Christopher J. Lortie}

\corrauthor[1]{Christopher J. Lortie}{\href{mailto:chris@christopherlortie.info}{\nolinkurl{chris@christopherlortie.info}}}

\affil[1]{NCEAS, UCSB, Santa Barbara, CA. 93101}


%
% \author[1]{First Author}
% \author[2]{Second Author}
% \affil[1]{Address of first author}
% \affil[2]{Address of second author}
% \corrauthor[1]{First Author}{f.author@email.com}

% 

\begin{abstract}
There is extensive support and choice in R to support meta-analyses. Two
common packages in the natural sciences include meta and metafor. Here,
a brief contrast of the strengths of each is described for the synthesis
scientist. Meta is a direct, intuitive choice for rapid implementation
of general meta-analytical statistics. Metafor is a comprehensive
package for analyses if the fit models are more complex. Both packages
provide estimates of heterogeneity, excellent visualization tools, and
functions to explore publication bias. Preference and critical outcomes
can facilitate choice between these two specific options. Nonetheless,
metafor has a steeper learning curve but greater rewards.
% Dummy abstract text. Dummy abstract text. Dummy abstract text. Dummy abstract text. Dummy abstract text. Dummy abstract text. Dummy abstract text. Dummy abstract text. Dummy abstract text. Dummy abstract text. Dummy abstract text.
\end{abstract}

\begin{document}

\flushbottom
\maketitle
\thispagestyle{empty}

\section*{Introduction}\label{introduction}
\addcontentsline{toc}{section}{Introduction}

Meta-analyses are common and powerful synthesis tools in science.
Typically in the natural sciences, meta-analyses are used as a mechanism
to describe and aggregate quantitative evidence from a set of
peer-reviewed, primary research publications (Christopher J. Lortie and
Bonte 2016, Nakagawa et al. (2017)). A meta-analysis in the natural
sciences, i.e.~outside the health sciences, is comprised of a formalized
systematic review and analysis of effect sizes and is termed a
meta-analysis when statistics examining intervention efficacy are
included (C. J. Lortie 2014). In other fields, the terms systematic
review and meta-analysis are used more interchangeable, and
meta-statistics are often done on compiled randomized controlled trials
or other relatively large datasets in addition to sets of data derived
from peer-reviewed publications. The derived data in the natural
sciences (as the intended primary audience here) routinely extracts only
the mean values or single point measures from publications (Stewart and
Schmid 2015). The synthesis statistics were commonly done using MetaWin
(Rosenberg, Adams, and Gurevitch 2000) or other GUI-based applications
for a number of years. More recently however, statistics in the fields
of ecology and evolution for instance have increasingly moved to the
programming language R (Lai et al. 2019), and synthesis statistics are
no exception. At least two R packages have risen to prominence for
general meta-analytical statistics in the natural sciences - namely meta
(Schwarzer 2019) and metafor (Viechtbauer 2017). Given that
meta-analyses are also increasingly published in these same fields
(Cadotte, Mehrkens, and Menge 2012, Christopher J. Lortie and Bonte
(2016)), a brief comment on the ecosystem of analytical choices that R
provides is beneficial and timely. We need synthesis to inform
evidence-based decisioning, and meta-analyses can be the primary tool if
aggregated primary datasets are unavailable. Furthermore even with
primary data in hand, data reduction to effect sizes within primary and
synthesis studies is a mechanism to illustrate differences and strength
of effects. These approaches provide the capacity for higher-order
analyses and reuse (Gerstner et al. 2017) suggesting that familiarity
with effect sizes is both germane and practical.

\section*{The R ecosystem for
meta-analyses}\label{the-r-ecosystem-for-meta-analyses}
\addcontentsline{toc}{section}{The R ecosystem for meta-analyses}

Like many fundamental challenges in science, the R developer community
provides potential solution sets distributed across multiple packages
for synthesis. Broadly speaking, alternative packages in R sometimes
examine an issue from different perspectives and provide unique
functions. In other instances, packages can be very similar or analogs
in terms of functionality and use conceptually aligned functions that
differ only in nomenclature or arguments. Scientific synthesists that
choose to do a meta-analyses in R have options. A total of 63 packages
associated with various aspects of conducting a meta-analysis have been
identified in a comprehensive review and typology of options (Polanin,
Hennessy, and Tanner-Smith 2016). Both meta (Schwarzer 2019) and metafor
(Viechtbauer 2017) are amongst 11 generic packages identified (Polanin,
Hennessy, and Tanner-Smith 2016). These two packages are analogs but
with different inherent workflows. There is also rmeta for simple fixed
and random effects meta-analyses (Lumley 2018), mada for diagnostics
(Doebler 2017), (Doebler 2017), netmeta for frequentist network meta
statistics (Rucker et al. 2019), and mvmeta for multivariate derived
data aggregations (Gasparrini 2018) to name a few options. The latter
three packages listed have distinct and specific niches for analysis
whilst meta and metafor overlap considerably. Consequently, a brief
contrast in facilitating choice between these two packages for the
general analyst is provided here.

\section*{Contrast of meta versus
metafor}\label{contrast-of-meta-versus-metafor}
\addcontentsline{toc}{section}{Contrast of meta versus metafor}

Meta is a well-maintained, recently updated CRAN R package (Version
4.9-4 updated on January 3, 2019) with 31 unique functions, 7 sample
datasets, and a reference manual. There is also an exceptional textbook
devoted to meta-analysis in R that focuses primarily on this package
(Schwarzer, Carpenter, and Rücker 2015). It is highly capable of
resolving most general meta-analytical challenges that an analyst will
face including the capacity to include Empirical Bayes estimators as
arguments in some functions, predictive meta-statistics, interaction
terms, meta-regression, and modifiers. Note that for some of these
methods, the rma.uni function is sourced internally from the package
metafor. This is intriguing but mostly opaque and inconsequential to the
user if she prefers the structure of the arguments within functions, the
semantics, or the workflow of the meta package. The primary strengths
include its direct and straightforward implementation with minimal
(source) lines of code to do an analysis. Provided one has secured the
derived data from the studies and organized into a dataframe with
vectors as each key argument within the main meta-model fitting
functions, statistics are simple. The type of response variable such as
mean, continuous, or rate is matched to a specific function call such as
metamean, metacont, or metarate. This is semantically intuitive and
encourages good thinking before statistics because it engenders
consideration of the data. The effect size calculation is included in
this main function and defaults return the most prevalent effect size
measure typically associated with those data, but it can also be
specified as an argument. The primary workflow is thus a single step if
the user elects to rely on the internal calculations provided in this
package. Exploration of the model is well articulated with funnel,
radial, and forest plots. Z-scores, significance tests, and
heterogeneity statistics are printed in the model summary. Publication
bias is also provided as a more in-depth function entitled metabias
within this package. There are two standout functions in this package.
The first is a function entitled metagen, and it is a backup,
multipurpose tool so to speak that fits a generic inverse variance
meta-analysis. This a handy tool for user-calculated effect size
measures or for exploration of statistical trends with reduced data
assumptions. In some fields, there are specific effect size estimates
that this function provides a robust, easy-to-fit capacity for
statistics. The second standout function is bubble.metareg for a quick,
visual exploration of the outcome of a meta-regression. It is useful in
contemporary data science to use visualization as a means to understand
data (C. Lortie 2017), but statistical packages do not always provide
the means to easily iterate between statistics or model fitting and
visualization. In summary, excepting unique data or statistical issues,
this package is directly implemented and effective.

Metafor is a more comprehensive package in many respects. This package
includes 74 functions, 35 datasets, a vignette (Viechtbauer 2010),
flowchart as secondary vignette
(\url{https://cran.r-project.org/web/packages/metafor/vignettes/diagram.pdf}),
and website (\url{http://www.metafor-project.org/doku.php}). The package
was last updated on CRAN in June of 2017 (Version 2.0-0). The text
`Meta-analysis with R' also describes implementation of this package
(Schwarzer, Carpenter, and Rücker 2015) but to a lesser extent than
meta. The depth of the package metafor provides greater capacities
relative to the meta package but does come at the expense of a steeper
initial learning curve. Completing a meta-analysis using this package
requires an additional step, i.e.~effect sizes must be calculated a
priori, not within the model fitting process. This is facilitated with
the standalone function escalc, and it can return a wide range of effect
sizes measures. Thus, the two-step process begins with firstly compiling
and aggregating the derived dataframe to an effect size table then
secondly fitting a model. The data structure is also a bit more fixed
for the model fitting, and the nomenclature for this subset of functions
is written to parallel more traditional general linear model fitting
from conventional statistics. This is both a strength and limitation
because one must plan the model to fit in advance and learn the function
and arguments but it is also an advantage as well because model
specification uses the familiar notation of tilde. Model fitting is
based on the type of model in the call such a random or fixed effects
and not on the type of the response data as in meta package. Here, it is
more akin to conventional general linear model fitting for those
familiar with these functions in R. If the model is more complex with
moderators, then this can be directly included in the model fit here via
a mods argument whereas in the meta package the model is updated with
moderators in a subsequent step. This suggests that if moderators or
covariates in the main model are likely relevant to the analyses, then
metafor is a strong starting point. The model summary also prints
z-scores, significance tests, and two sets of heterogeneity estimates.
Forest and radial plots are also provided as additional functions.
Publication bias statistical estimator functions include trimfill and
ranktest. Standout elements of this package are primarily associated
with enhanced model fitting capacities such as the function fitstats
that provides log-likelihood estimates and AIC or BIC scores on
meta-analysis objects. This package requires a focus on model fitting,
and while there is additional effort in specifying the data at the onset
of the workflow, the rewards in subsequent tools to handle models are
significant.

\section*{Conclusions}\label{conclusions}
\addcontentsline{toc}{section}{Conclusions}

Statistics are sometimes about preferences and thinking styles (Hector
2017), and scientific synthesis is both an art and a science
(Christopher J. Lortie and Bonte 2016). Trade-offs are also common in
adopting one ecosystem, analysis tool, or specific package for data
wrangling and analyses. If more rapid, less specified, general
meta-analyses are the goal -- the package meta is a direct means to an
end. Moderators are added post hoc in additional, update model steps,
but the first model fit is a single, intuitive process. Meta-regression
is viable and interaction terms can be included. The generic
meta-analysis function is a superb tool. Metafor requires the effect
size compilation a priori and is thus a bit more coding to prepare for
the meta-model. However, deeper and more complex model fits are inherent
in the semantics of these functions. If the synthesist does have not
effect size measure in hand or wishes to calculate effect sizes measures
but not for meta-models, the escalc function is invaluable in this
package. In summary, both packages provide the capacity for basic and
advanced meta-analyses but more advanced modelling is likely worth the
commitment to metafor.

\section*{Acknowledgments}\label{acknowledgments}
\addcontentsline{toc}{section}{Acknowledgments}

CJL is a Senior Research Associate at NCEAS and in-kind computational
support is provided.

\section*{References}\label{references}
\addcontentsline{toc}{section}{References}

\hypertarget{refs}{}
\hypertarget{ref-RN2189}{}
Cadotte, M.W., L.R. Mehrkens, and D.N.L. Menge. 2012. ``Gauging the
Impact of Meta-Analysis on Ecology.'' Journal Article.
\emph{Evolutionary Ecology} 26: 1153--67.

\hypertarget{ref-RN6196}{}
Doebler, P. 2017. ``Mada.'' Journal Article. \emph{CRAN}, Version 0.5.8.

\hypertarget{ref-RN6198}{}
Gasparrini, A. 2018. ``Mvmeta.'' Journal Article. \emph{CRAN}, Version
0.4.11.

\hypertarget{ref-RN4835}{}
Gerstner, Katharina, David Moreno-Mateos, Jessica Gurevitch, Michael
Beckmann, Stephan Kambach, Holly P. Jones, and Ralf Seppelt. 2017.
``Will Your Paper Be Used in a Meta-Analysis? Make the Reach of Your
Research Broader and Longer Lasting.'' Journal Article. \emph{Methods in
Ecology and Evolution} 8 (6): 777--84.
doi:\href{https://doi.org/10.1111/2041-210X.12758}{10.1111/2041-210X.12758}.

\hypertarget{ref-RN6087}{}
Hector, A. 2017. \emph{The New Statistics with R}. Book. Oxford: Oxford
University Press.

\hypertarget{ref-RN6098}{}
Lai, Jiangshan, Christopher J. Lortie, Robert A. Muenchen, Jian Yang,
and Keping Ma. 2019. ``Evaluating the Popularity of R in Ecology.''
Journal Article. \emph{Ecosphere} 10 (1): e02567.
doi:\href{https://doi.org/10.1002/ecs2.2567}{10.1002/ecs2.2567}.

\hypertarget{ref-RN3216}{}
Lortie, C. J. 2014. ``Formalized Synthesis Opportunities for Ecology:
Systematic Reviews and Meta-Analyses.'' Journal Article. \emph{Oikos}
123: 897--902.

\hypertarget{ref-RN4510}{}
Lortie, Christopher. 2017. ``R for Data Science.'' Journal Article.
\emph{Journal of Statistical Software} 77: 1--3.
doi:\href{https://doi.org/10.18637/jss.v077.b01}{10.18637/jss.v077.b01}.

\hypertarget{ref-RN3629}{}
Lortie, Christopher J., and Dries Bonte. 2016. ``Zen and the Art of
Ecological Synthesis.'' Journal Article. \emph{Oikos} 125 (3): 285--87.
doi:\href{https://doi.org/10.1111/oik.03161}{10.1111/oik.03161}.

\hypertarget{ref-RN6195}{}
Lumley, T. 2018. ``Rmeta.'' Journal Article. \emph{CRAN}, Version 3.0.

\hypertarget{ref-RN4850}{}
Nakagawa, Shinichi, Daniel W. A. Noble, Alistair M. Senior, and
Malgorzata Lagisz. 2017. ``Meta-Evaluation of Meta-Analysis: Ten
Appraisal Questions for Biologists.'' Journal Article. \emph{BMC
Biology} 15 (1): 18.
doi:\href{https://doi.org/10.1186/s12915-017-0357-7}{10.1186/s12915-017-0357-7}.

\hypertarget{ref-RN6178}{}
Polanin, Joshua R., Emily A. Hennessy, and Emily E. Tanner-Smith. 2016.
``A Review of Meta-Analysis Packages in R.'' Journal Article.
\emph{Journal of Educational and Behavioral Statistics} 42 (2): 206--42.
doi:\href{https://doi.org/10.3102/1076998616674315}{10.3102/1076998616674315}.

\hypertarget{ref-RN6194}{}
Rosenberg, Michael, Dean Adams, and Jessica Gurevitch. 2000.
\emph{MetaWin: Statistical Software for Meta-Analysis. Version 2.0}.
Book.

\hypertarget{ref-RN6197}{}
Rucker, G., U. Krahn, J. Konig, O. Efthimious, and G. Schwarzer. 2019.
``Netmeta.'' Journal Article. \emph{CRAN} 1.0.1: Version 1.0.1.

\hypertarget{ref-RN6176}{}
Schwarzer, G. 2019. ``Meta.'' Journal Article. \emph{CRAN}, Version
4.94.

\hypertarget{ref-RN6199}{}
Schwarzer, G., J.R. Carpenter, and G. Rücker. 2015. \emph{Meta-Analysis
with R}. Book. Switzerland: Springer.

\hypertarget{ref-RN4861}{}
Stewart, G. B., and C. H. Schmid. 2015. ``Lessons from Meta-Analysis in
Ecology and Evolution: The Need for Trans-Disciplinary Evidence
Synthesis Methodologies.'' Journal Article. \emph{Research Synthesis
Methods} 6: 109--10.
doi:\href{https://doi.org/10.1002/jrsm.1152}{10.1002/jrsm.1152}.

\hypertarget{ref-RN4896}{}
Viechtbauer, W. 2010. ``Conducting Meta-Analyses in R with the Metafor
Package.'' Journal Article. \emph{J Stat Software} 36.
doi:\href{https://doi.org/10.18637/jss.v036.i03}{10.18637/jss.v036.i03}.

\hypertarget{ref-RN6175}{}
---------. 2017. ``Metafor.'' Journal Article. \emph{CRAN}, Version 2.0.



\end{document}
